\section{A new upper bound for the de Bruijn-Newman constant}

In this section we prove Theorem \ref{new-upper}.  As stated in the introduction, it suffices to verify the conditions (i), (ii), (iii) of Theorem \ref{ubc-0} $t_0 := 0.2$, $X := 6 \times 10^{10} + 83952$, and $y_0 = 0.2$.  

Claim (i) is immediate from the result of Platt \cite{platt} that all the non-trivial zeroes of $\zeta$ with imaginary part between $0$ and $3.06 \times 10^{10}$ lie on the critical line $\{ \mathrm{Re}(s) = 1/2\}$.  For the remaining claims (ii), (iii), one has to verify $H_t(x+iy) \neq 0$ for various $(x,y,t)$ in the region \eqref{region}.
To show that $H_t(x+iy) \neq 0$, it suffices from Theorem \ref{eff} to show that
\begin{equation}\label{bbb}
|f_t(x+iy)| > e_A + e_B + e_{C,0} 
\end{equation}
where
\begin{align*}
f_t(x+iy) &\coloneqq \sum_{n=1}^N \frac{b_n^t}{n^{s_*}} + \gamma \sum_{n=1}^N n^y \frac{b_n^t}{n^{\overline{s_*} + \kappa}}\\
b_n^t &\coloneqq \exp( \frac{t}{4} \log^2 n),\\
N &\coloneqq \lfloor \sqrt{\frac{x}{4\pi} + \frac{t}{16}} \rfloor \leq [\frac{x}{4\pi} (1 + \frac{\pi t}{4x})]^{1/2}
\end{align*}
so in particular
\begin{equation}\label{logx}
 \log \frac{x}{4\pi} \geq 2 \log N - \log(1 + \frac{\pi t}{4x}),
\end{equation}
$\kappa, s_*, \gamma$ are as in Theorem \ref{eff}, and $e_A, e_B, e_{C_0}$ are bounded by the bounds in that theorem.

Note that if $x \geq 6 \times 10^{10} + 83952$ then $N \geq 69098$.  We calculate a somewhat crude upper bound for the right-hand side of \eqref{bbb}:

{\bf the estimate below needs to be redone for the new range of parameters}

\begin{lemma}\label{lac}  If $x \geq 5 \times 10^9$, $0 \leq t \leq 0.2$, and $0.4 \leq y \leq 1$, then
$$ e_A + e_B + e_{C,0} \leq 4.37 \times 10^{-4} + 6.53 \times 10^{-10} F_{N,t}(0.7 + \frac{t}{4} \log \frac{x}{4\pi}).$$
where
\begin{equation}\label{fnt-def}
 F_{N,t}( \sigma ) := \sum_{n=1}^N \frac{b_n^t}{n^\sigma}.
\end{equation}
\end{lemma}

\begin{proof}
From Theorem \ref{eff} we have
\begin{equation}\label{eaeb-bound}
 e_A + e_B \leq (e^{\delta_1}-1) (F_{N,t}(\mathrm{Re}(s_*)) + |\gamma| F_{N,t}( \mathrm{Re}(s_*) - y - |\kappa| ) )
\end{equation}
where
$$ \delta_1 := \frac{\frac{t^2}{16} \log^2 \frac{x}{4\pi} + 0.626}{x-6.66}.$$
From Lemma \ref{elem-lem}(vi), the quantity $\delta_1$ is monotone decreasing in $x$ in the region \eqref{region}.  Thus we have
\begin{equation}\label{delta1-bound}
 \delta_1 \leq \frac{\frac{t_0^2}{16} \log^2 \frac{x_0}{4\pi} + 0.626}{x_0-6.66}
\end{equation}
whenever $x \geq x_0 \geq 200$ and $0 \leq t \leq t_0$.  Substituting $t_0 = 0.2$, $x_0 = 5 \times 10^9$ we conclude that
$$  \delta_1 \leq 3.22 \times 10^{-10}$$
and hence
$$ e^{\delta_1} - 1 \leq 3.23 \times 10^{-10}.$$
Also, from Theorem \ref{eff} and \eqref{fnt-def} we can bound
\begin{align*}
|\gamma| F_{N,t}( \mathrm{Re}(s_*) - y - |\kappa| ) &\leq |\gamma| N^y N^{|\kappa|} F_{N,t}( \mathrm{Re}(s_*) ) \\
&\leq \exp( 0.02 y + y (\log N - \frac{1}{2} \log \frac{x}{4\pi}) + \frac{ty}{2(x-6)} \log N ) F_{N,t}( \mathrm{Re}(s_*) )  \\
&\leq \exp( 0.02 y + (y + \frac{ty}{2(x-6)} \frac{1}{2} \log(1 + \frac{\pi t}{4x}) + \frac{ty}{4(x-6)} \log \frac{x}{4\pi} ) F_{N,t}( \mathrm{Re}(s_*) ).
\end{align*}
For $y \leq 1$, $0 \leq t \leq 0.2$, and $x \geq 5 \times 10^9$ we see from Lemma \ref{elem-lem}(vi) that
$$ \frac{ty}{4(x-6)} \log \frac{x}{4\pi} \leq \frac{0.2}{4(5 \times 10^9-6)}\log \frac{5 \times 10^9}{4\pi} \leq 1.99 \times 10^{-10}$$
and
$$ (y + \frac{ty}{2(x-6)} \frac{1}{2} \log(1 + \frac{\pi t}{4x}) \leq 1.58 \times 10^{-11}$$
and thus
\begin{equation}\label{gafn}
 |\gamma| F_{N,t}( \mathrm{Re}(s_*) - y - |\kappa| ) \leq 1.021  F_{N,t}( \mathrm{Re}(s_*) ).
\end{equation}
Thus
$$  e_A + e_B \leq 6.53 \times 10^{-10} F_{N,t}( \mathrm{Re}(s_*) ).$$
From Proposition \ref{estimates}(ii) (and the observation that $1-3y+\frac{4y(1+y)}{x^2}$ is negative when $y \geq 0.4$ and $x \geq 5 \times 10^9$) we have
$$ \mathrm{Re}(s_*) \geq 0.7 + \frac{t}{4} \log \frac{x}{4\pi}.$$
Since $F_{N,t}(\sigma)$ is non-increasing in $\sigma$, we conclude
$$  e_A + e_B \leq 6.53 \times 10^{-10} F_{N,0.2}( 0.7 + \frac{t}{4} \log \frac{x}{4\pi} ).$$

From Proposition \ref{estimates}(vi) one has
$$ e_{C,0} \leq \left(\frac{x}{4\pi}\right)^{-\frac{1+y}{4}} \exp\left( - \frac{t}{16} \log^2 \frac{x}{4\pi} + \frac{3 |\log \frac{x}{4\pi} + i \frac{\pi}{2}|+3.58}{x-8.52} \right) \left(1 + \frac{1.24 \times (3^y+3^{-y})}{N-0.125} + \frac{6.72}{x-6.66}\right).$$
From Lemma \ref{elem-lem}(vi), the quantities $\frac{\log^2 \frac{x}{4\pi} + \frac{\pi^2}{4}}{(x-8.52)^2}$ is monotone decreasing in $x$ in \eqref{region}, hence $\frac{|\log \frac{x}{4\pi} + i \frac{\pi}{2}|}{x-8.52}$ is also monotone decreasing.  Also the expression is monotone decreasing in $y$. We conclude that
\begin{equation}\label{ec0-bound}
 e_{C,0} \leq \left(\frac{x_0}{4\pi}\right)^{-\frac{1+y_0}{4}} \exp\left( - \frac{t}{16} \log^2 \frac{x_0}{4\pi} + \frac{3 |\log \frac{x_0}{4\pi} + i \frac{\pi}{2}|+3.58}{x_0-8.52} \right) \left(1 + \frac{1.24 \times (3^{y_0}+3^{-y_0})}{N_0-0.125} + \frac{6.72}{x_0-6.66}\right)
\end{equation}
whenever $x \geq x_0 \geq 200$, $y \geq y_0 > 0$, and $N \geq N_0$.  Setting $x_0 = 5 \times 10^9$, $y_0 = 0.4$, and $N_0 = 19947$ and using $t \geq 0$ we conclude that
From \eqref{ec0-bound} one has
\begin{align*}
e_{C,0} &\leq \left(\frac{5 \times 10^9}{4\pi}\right)^{-\frac{1+0.4}{4}} \exp\left( - \frac{t}{16} \log^2 \frac{5 \times 10^9}{4\pi} + \frac{3 |\log \frac{5 \times 10^9}{4\pi} + i \frac{\pi}{2}|+3.58}{5 \times 10^9-8.52} \right) \left(1 + \frac{1.24 \times (3^{0.4}+3^{-0.4})}{19947-0.125} + \frac{6.92}{5 \times 10^9-6.66}\right)
&\leq 4.37 \times 10^{-4}.
\end{align*}
The claim follows.
\end{proof}


It now suffices to establish the following claims:

\begin{itemize}
\item[(i)]  \eqref{bbb} holds when $6 \times 10^{10} + 83952 \leq x \leq 6 \times 10^{10} + 83953$, $0 \leq t \leq 0.2$, and $y \geq 0.2$.
\item[(ii)]  \eqref{bbb} holds when $x \geq 6 \times 10^{10} + 83952$, $69098 \leq N \leq ???$, $t = 0.2$, and $y \geq 0.2$.
\item[(iii)]  \eqref{bbb} holds when $??? \leq N \leq 1.5 \times 10^6$, $t = 0.2$, and $y \geq 0.2$.
\item[(iv)]  \eqref{bbb} holds when $N \geq 1.5 \times 10^6$, $t = 0.2$, and $y \geq 0.2$.
\end{itemize}


We begin with claim (i).  We need some derivative estimates on the quantity $f_t(x+iy )$.

\begin{lemma} In the region \eqref{region}, and away from the jump discontinuities of $N$, we have
$$ |\frac{\partial f_t}{\partial x}| = |\frac{\partial f_t}{\partial y}| \leq  \sum_{n=1}^N \frac{b_n^t}{n^{\mathrm{Re}(s_*)}} (\frac{\log n}{2} + \frac{t \log n}{4(x-6)}) + |\gamma| N^{|\kappa|} \sum_{n=1}^N \frac{b_n^t n^{y} }{n^{\mathrm{Re}(s_{*})}}
( \frac{t \log n}{4(x-6)} + (\log \frac{|1+y+ix|}{4\pi} + \pi + \frac{3}{x}) (\frac{1}{2} + \frac{t}{4(x-6)})) $$
and
\begin{align*} |\frac{\partial f_t}{\partial t}| &\leq \sum_{n=1}^N \frac{b_n^t}{n^{\mathrm{Re} s_*}} (\frac{1}{4} \log n \log \frac{x}{4\pi n} + \frac{\pi}{8} \log n + \frac{2 \log n}{x-6}) \\
&\quad + |\gamma| N^{|\kappa|} \sum_{n=1}^N \frac{b_n^t n^y}{n^{\mathrm{Re}(s_{*})}}
(\frac{1}{4} \log n \log \frac{x}{4\pi n} + \frac{\pi}{8} \log n + + \frac{2 \log n}{x-6} + \frac{1}{4} (\frac{\pi}{2} + \frac{8}{x-6}) (\log \frac{x}{4\pi} + \frac{8}{x-6})).
\end{align*}
\end{lemma}

\begin{proof}
We begin with the first estimate.  Write 
$$ s_{**} \coloneqq \overline{s_*} - y + \kappa = \frac{1-y+ix}{2} + \frac{t}{2} \alpha(\frac{1-y+ix}{2})$$
then
\begin{equation}\label{ftne}
f_t = \sum_{n=1}^N \frac{b_n^t}{n^{s_*}} + \gamma \sum_{n=1}^N \frac{b_n^t}{n^{s_{**}}}.
\end{equation}
One can check that $s_*, s_{**}, \gamma$ are holomorphic functions of $x+iy$, hence by the Cauchy-Riemann equations
$$ |\frac{\partial f_t}{\partial x}| = |\frac{\partial f_t}{\partial y}|.$$
By the product and chain rules, we may calculate
$$ 
\frac{\partial f_t}{\partial x} = - \sum_{n=1}^N \frac{b_n^t}{n^{s_*}} \frac{\partial s_*}{\partial x} \log n + \gamma \sum_{n=1}^N \frac{b_n^t}{n^{s_{**}}}
( \frac{\partial}{\partial x} \log \gamma - \frac{\partial s_{**}}{\partial x} \log n).$$
From \eqref{sn-def}, \eqref{alpha-deriv-bound} we have
\begin{align*}
 \frac{\partial s_*}{\partial x} &= -\frac{i}{2} - \frac{it}{4} \alpha'(\frac{1-y+ix}{2}) \\
&= -\frac{i}{2} + O_{\leq}( \frac{t}{4(x-6)} ).
\end{align*}
Similarly we have
$$ \frac{\partial s_{**}}{\partial x} = \frac{i}{2} + O_{\leq}( \frac{t}{4(x-6)} ).$$
Writing $s = \frac{1-y+ix}{2}$, we have from \eqref{lambda-def}, \eqref{Mt-def} that
$$ \log \gamma = \frac{t}{4} (\alpha(s)^2 - \alpha(1-s)^2) + \log M_0(s) - \log M_0(1-s) $$
and hence by \eqref{alpha-def}
$$ \frac{\partial \gamma}{\partial x} = \frac{it}{4} (\alpha(s) \alpha'(s) + \alpha(1-s) \alpha'(1-s))
+ \frac{i}{2} \alpha(s) + \frac{i}{2} \alpha(1-s).$$
From the triangle inequality and \eqref{alpha-deriv-bound}, we thus have
$$ 
|\frac{\partial f_t}{\partial x}| \leq \sum_{n=1}^N \frac{b_n^t}{n^{\mathrm{Re}(s_*)}} (\frac{\log n}{2} + \frac{t \log n}{4(x-6)}) + |\gamma| \sum_{n=1}^N \frac{b_n^t}{n^{\mathrm{Re}(s_{**})}}
( \frac{t \log n}{4(x-6)} + \frac{|\alpha(s) + \alpha(1-s) - \log n|}{2} + \frac{t (|\alpha(s)| + |\alpha(1-s)|)}{4(x-6)}).$$
We have from \eqref{alpha-form} that
$$ |\alpha(s)|, |\alpha(s) - \frac{1}{2} \log n| \leq \frac{1}{2} \log \frac{|1-y+ix|}{4\pi} + \frac{\pi}{2} + \frac{3}{2x} $$
since $n \leq N \leq \frac{x}{4\pi} \leq \frac{|1-y+ix|}{4\pi}$.  Similarly
$$ |\alpha(1-s)|, |\alpha(1-s) - \frac{1}{2} \log n| \leq \frac{1}{2} \log \frac{|1+y+ix|}{4\pi} + \frac{\pi}{2} + \frac{3}{2x} $$
and thus
$$ |\alpha(s)+\alpha(1-s)|, |\alpha(s)+\alpha(1-s)-\log n| \leq \log \frac{|1+y+ix|}{4\pi} + \pi + \frac{3}{x}.$$
Writing $\mathrm{Re}(s_{**}) = \mathrm{Re}(s_*) - y + \mathrm{Re}(\kappa)$, we then have the first estimate.

Now we estimate the time derivative.  Since
\begin{align*}
 \frac{\partial}{\partial t} \log b_n^t &= \frac{1}{4} \log^2 n \\
 \frac{\partial}{\partial t} s_* &= \frac{1}{2} \alpha(\frac{1+y-ix}{2}) \\
 \frac{\partial}{\partial t} s_{**} &= \frac{1}{2} \alpha(\frac{1-y+ix}{2}) \\
 \frac{\partial}{\partial t} \log \gamma &= \frac{1}{4} (\alpha(\frac{1-y+ix}{2})^2 - \alpha^2(\frac{1+y-ix}{2}))
\end{align*}
we see from differentiating \eqref{ftne} that, we obtain
$$ \frac{\partial f_t}{\partial t} = \sum_{n=1}^N \frac{b_n^t}{n^{s_*}} (\frac{\log^2 n}{4} - \frac{\alpha(\frac{1+y-ix}{2})}{2} \log n) 
+ \gamma \sum_{n=1}^N \frac{b_n^t}{n^{s_{**}}}
(\frac{\log^2 n}{4} - \frac{\alpha(\frac{1-y+ix}{2})}{2} \log n + \frac{1}{4} (\alpha(\frac{1-y+ix}{2})^2 - \alpha^2(\frac{1+y-ix}{2}))).$$
From \eqref{alpha-deriv-bound}, \eqref{alpha-form} we have
\begin{align*}
 \alpha(\frac{1 \pm y+ix}{2}) &= \alpha(\frac{ix}{2}) + O_{\leq}( \frac{1}{x-6} ) \\
&= \frac{1}{2} \log \frac{x}{4\pi} + \frac{\pi i}{4} + O_{\leq}( \frac{4}{x-6} ) 
\end{align*}
and hence (since $\alpha = \alpha^*$)
$$ \alpha(\frac{1 \pm y-ix}{2}) = \frac{1}{2} \log \frac{x}{4\pi} - \frac{\pi i}{4} + O_{\leq}( \frac{4}{x-6} ) $$
so in particular
$$ \alpha(\frac{1-y+ix}{2}) - \alpha(\frac{1+y-ix}{2}) = \frac{\pi i}{2} + O_{\leq}( \frac{8}{x-6} )$$
and
$$ \alpha(\frac{1-y+ix}{2}) + \alpha(\frac{1+y-ix}{2}) = \log \frac{x}{4\pi} + O_{\leq}( \frac{8}{x-6} )$$
so that
$$ |\alpha(\frac{1-y+ix}{2})^2 - \alpha^2(\frac{1+y-ix}{2}))| \leq (\frac{\pi}{2} + \frac{8}{x-6}) (\log \frac{x}{4\pi} + \frac{8}{x-6}).$$
We conclude from the triangle inequality that
\begin{align*}
 |\frac{\partial f_t}{\partial t}| &\leq \sum_{n=1}^N \frac{b_n^t}{n^{\mathrm{Re} s_*}} (\frac{1}{4} \log n \log \frac{x}{4\pi n} + \frac{\pi}{8} \log n + \frac{2 \log n}{x-6}) \\
&+ |\gamma| \sum_{n=1}^N \frac{b_n^t}{n^{\mathrm{Re}(s_{**})}}
(\frac{1}{4} \log n \log \frac{x}{4\pi n} + \frac{\pi}{8} \log n + \frac{2 \log n}{x-6} + \frac{1}{4} (\frac{\pi}{2} + \frac{8}{x-6}) (\log \frac{x}{4\pi} + \frac{8}{x-6}))
\end{align*}
giving the second claim.
\end{proof}


...


Now we prove claim (ii).

...

Now we prove claim (iii).

We attempt here to find N dependent positive lower bounds for $|f_{t}(x+iy)|$ or corresponding mollified sums through an incremental approach,

$ |f_{N+1}| >= |f_{N}| - \sum$ (additional terms correesponding to N+1), 
with the bound for $|f_{N_{min}}|$ computed as in someref or someref.

If the incremental bound goes below a positive threshold at N, we reset $N_{min}$ to N and restart the process, generating a sawtooth pattern.

For the Triangle inequality bound,

$|f_{N+1}| >= |f_{N}| - \sum\limits_{n=DN+1}^{DN+D} (\frac{\sum\limits_{1<=d<=D:d|n} \lambda_{d} b_{n/d}^{t} (1 + |\gamma|n^y))}{n^\sigma}$

For the Lemma inequality bound, we first make some observations.

1) $\frac{1-a_1}{1+a_1} = \frac{1-|\gamma|}{1+|\gamma|}$ increases with N.

2) If $\beta_{n} \alpha_{n} >= 0$ i.e. $\beta_{n}$ and $\alpha_{n}$ do not have oppposite signs,

$|\beta_{n} + \alpha_{n}| >= |\beta_{n} - \alpha_{n}|$
hence $\max (|\beta_{n} + \alpha_{n}|, |\beta_{n} - \alpha_{n}|) = |\beta_{n} + \alpha_{n}|$

Now, $\beta_{n} = \sum\limits_{1<=d<=D:d|n} \lambda_{d} b_{n/d}^{t} = b_{n}^{t} \sum\limits_{1<=d<=D:d|n} \lambda_{d} ({\frac{d}{n^2}})^{(t/4) \log d}$

and, $\alpha_{n} = \sum\limits_{1<=d<=D:d|n} \lambda_{d} a_{n/d}^{t} = b_{n}^{t}|\gamma|n^y \sum\limits_{1<=d<=D:d|n} \lambda_{d} ({\frac{d}{n^2}})^{(t/4) \log d} \frac{1}{d^y}$

Hence, subject to condition 2) and replacing ,


$|f_{N+1}| >= |f_{N}| - \sum\limits_{n=DN+1}^{DN+D} (\sum\limits_{1<=d<=D:d|n} \lambda_{d} b_{n/d}^{t} (1 + |\gamma|n^y))$


...

{\bf all numbers need to be recalculated here}

Finally, we prove claim (iv).  In this regime we have
$$ x \geq 4\pi N_0^2 - \frac{\pi t}{4} \geq 1.13 \times 10^{12}.$$
We use the triangle inequality bound
\begin{align*}
|X| &\geq \left( 2 - F_{N,t}(\mathrm{Re}(s_*)) - |\gamma| F_{N,t}(\mathrm{Re}(s_*) - y) \right)_+ \\
&\geq \left( 2 - F_{N,0.2}(0.7 + \frac{1}{20} \log \frac{x}{4\pi}) - |\gamma| F_{N,0.2}(0.3 + \frac{1}{20} \log \frac{x}{4\pi}) \right)_+ 
\end{align*}
so to prove \eqref{bbb} here, it suffices by Lemma \ref{lac} to show that
\begin{equation}\label{fna}
F_{N,0.2}(0.7 + \frac{1}{20} \log \frac{x}{4\pi}) + |\gamma| F_{N,0.2}( 0.3 + \frac{1}{20} \log \frac{x}{4\pi} )  \leq 2 - 4.38 \times 10^{-4}.
\end{equation}
From Proposition \ref{estimates}(i) we have
$$ |\gamma| \leq e^{0.0016} \left( \frac{x}{4\pi} \right)^{-0.2}.$$
We can then bound the left-hand side of \eqref{fna} by $Q_1+Q_2+Q_3$, where
\begin{align*}
Q_1 &\coloneqq F_{N_0,0.2}(0.7 + \frac{1}{20} \log \frac{x}{4\pi}) \\
Q_2 &\coloneqq e^{0.0016} \left( \frac{x}{4\pi} \right)^{-0.2} F_{N_0,0.2}( 0.3 + \frac{1}{20} \log \frac{x}{4\pi} ) \\
Q_3 &\coloneqq \sum_{N_0 < n \leq N} \frac{\exp( \frac{1}{20} \log^2 n)}{n^{0.7 + \frac{1}{20} \log \frac{x}{4\pi}}} (1 + e^{0.0016} (\frac{x}{4\pi n^2})^{-0.2}).
\end{align*}
We have
\begin{align*}
Q_1 &\leq F_{N_0,0.2}(0.7 + \frac{1}{20} \log \frac{1.13 \times 10^{12}}{4\pi}) \\
&\leq 1.898
\end{align*}
and
\begin{align*}
Q_2 &\leq e^{0.0016} \left( \frac{1.13 \times 10^{12}}{4\pi} \right)^{-0.2}  F_{N_0,0.2}(0.3 + \frac{1}{20} \log \frac{1.13 \times 10^{12}}{4\pi}) \\
&\leq 0.047.
\end{align*}
To estimate $Q_3$, we first observe that
\begin{align*}
\frac{x}{4\pi n^2} &\geq \frac{x}{4\pi N^2} \\
&\geq \left(1 + \frac{\pi t}{4x}\right)^{-1} \\
&\geq 1 - 1.4 \times 10^{-13}
\end{align*}
and hence
$$ 1 + e^{0.0016} (\frac{x}{4\pi n^2})^{-0.2} \leq 2.0017.$$
Also from \eqref{logx} we have
$$ \frac{1}{n^{\frac{1}{20} \log \frac{x}{4\pi}}} \leq \frac{1}{n^{\frac{1}{20} \log N^2}} \exp(\log(1 + \frac{\pi t}{4x}) \log N).$$
Using Proposition \ref{elem-lem}(vi) we can easily bound
$$ \log(1 + \frac{\pi t}{4x}) \log N \leq 10^{-6}$$
(say), and hence
$$ Q_3 \leq 2.0018 \sum_{N_0 < n \leq N} \frac{\exp( \frac{1}{20} \log^2 n)}{n^{0.7 + \frac{1}{20} \log N^2}}.$$
The function $n \mapsto \frac{\exp( \frac{1}{20} \log^2 n)}{n^{0.7 + \frac{1}{20} \log N^2}}$ is decreasing for $1 \leq n < N$, so by the integral test we have
$$ Q_3 \leq 2.0018 \int_{N_0}^N \frac{\exp( \frac{1}{20} \log^2 a)}{a^{0.7 + \frac{1}{20} \log N^2}}\ da.$$
Making the change of variables $a = e^u$, this becomes
$$ Q_3 \leq 2.0018 \int_{\log N_0}^{\log N} \exp( 0.3 u + \frac{1}{20} (u^2 - 2u \log N) )\ du.$$
The quadratic expression $- 0.3 u + \frac{1}{20} (u^2 - 2u \log N)$ is convex in $u$, and is thus bounded by its maximum at the endpoints; thus
$$ Q_3 \leq 2.0018 (\log \frac{N}{N_0}) \exp( \max( 0.3 \log N_0 -\frac{1}{20} \log N_0 \log \frac{N^2}{N_0}, 0.3 \log N - \frac{1}{20} \log^2 N ) ).$$
Writing $N = e^b N_0$ for some $b>0$, this bound can be rewritten as 
$$ Q_3 \leq \frac{2.0018}{N_0^{\frac{1}{20} \log N_0 - 0.3}} \exp( \log b + \frac{1}{20} ( -2b \log N_0 + (6b-b^2)_+ ) ).$$
A routine numerical maximisation shows that
$$ \log b + \frac{1}{20} ( -2b \log N_0 + (6b-b^2)_+ ) \leq -1.009$$
and hence
$$ Q_3 \leq 0.0113.$$
Combining these bounds we obtain \eqref{fna}.


