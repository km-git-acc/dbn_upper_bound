\section{Elementary estimates}

In order to explicitly estimate various error terms arising in the proof of Theorem \ref{eff}, we will need the following elementary estimates:

\begin{lemma}[Elementary estimates]\label{elem-lem} Let $x > 0$.
\begin{itemize}
\item[(i)] If $a > 0$ and $b \geq 0$ are such that $x > b/a$, then
$$O_{\leq}\left(\frac{a}{x}\right) + O_{\leq}\left( \frac{b}{x^2}\right ) = O_{\leq}\left( \frac{a}{x-b/a} \right).$$
More generally, if $a > 0$ and $b,c \geq 0$ are such that $x > b/a, \sqrt{c/a}$, then
$$O_{\leq}\left(\frac{a}{x}\right) + O_{\leq}\left( \frac{b}{x^2} \right) + O_{\leq}\left( \frac{c}{x^3}\right) = O_{\leq}\left( \frac{a}{x-\max(b/a,\sqrt{c/a})} \right).$$
\item[(ii)]  If $x > 1$, then
$$\log\left(1 + O_{\leq}\left(\frac{1}{x}\right) \right) = O_{\leq}\left(\frac{1}{x-1}\right).$$
or equivalently
$$1 + O_{\leq}\left(\frac{1}{x}\right) = \exp\left( O_{\leq}\left(\frac{1}{x-1}\right) \right).$$
\item[(iii)]  If $x > 1/2$, then
$$\exp\left( O_{\leq}\left(\frac{1}{x}\right) \right) = 1 + O_{\leq}\left( \frac{1}{x-0.5} \right).$$
\item[(iv)]  We have
$$ \exp\left(O_{\leq}(x)\right) = 1 + O_{\leq}(e^x-1).$$
\item[(v)] If $z$ is a complex number with $|\mathrm{Im}(z)| \geq 1$ or $\mathrm{Re} z \geq 1$, then
$$ \Gamma(z) = \sqrt{2\pi} \exp\left( \left(z-\frac{1}{2}\right) \log z - z + O_{\leq}\left( \frac{1}{12(|z| - 0.33)} \right)\right).$$
\item[(vi)] If $a,b > 0, y \geq 0$ and $x \geq x_0 \geq \exp(a/b)$ and $x_0 > c \geq 0$, then
$$\frac{\log^a|x+iy|}{(x-c)^b} \leq \frac{\log^a |x_0+iy|}{(x_0-c)^b}.$$
\end{itemize}
\end{lemma}

\begin{proof}  Claim (i) follows from the geometric series formula
$$ \frac{a}{x-t} = \frac{a}{x} + \frac{at}{x^2} + \frac{at^2}{x^3} + \dots$$
whenever $0 \leq t < x$.

For Claim (ii), we use the Taylor expansion of the logarithm to note that
$$\log\left( 1 + O_{\leq}\left(\frac{1}{x}\right) \right) = O_{\leq}\left(\frac{1}{x} + \frac{1}{2x^2} + \frac{1}{3x^3} + \dots\right)$$
which on comparison with the geometric series formula
$$\frac{1}{x-1} = \frac{1}{x} + \frac{1}{x^2} + \frac{1}{x^3} + \dots$$
gives the claim.  Similarly for Claim (iii), we may compare the Taylor expansion
$$\exp\left( O_{\leq}\left(\frac{1}{x}\right) \right) = 1 + O_{\leq}\left(\frac{1}{x} + \frac{1}{2! x^2} + \frac{1}{3! x^3} + \dots\right)$$
with the geometric series formula
$$ \frac{1}{x-0.5} = \frac{1}{x} + \frac{1}{2x^2} + \frac{1}{2^2 x^3} + \dots$$
and note that $k! \geq 2^k$ for all $k \geq 2$.

Claim (iv) follows from the trivial identity $e^x = 1 + (e^x-1)$ and the elementary inequality $e^{-x} \geq 1 - (e^x-1)$.
For Claim (v), we may use the functional equation $\Gamma = \Gamma^*$ to assume that $\mathrm{Im}(z) \geq 0$.  From the work of Boyd \cite[(1.13), (3.1), (3.14), (3.15)]{boyd} we have the effective Stirling approximation
$$ \Gamma(z) = \sqrt{2\pi} \exp\left( \left(z-\frac{1}{2}\right) \log z - z \right) \left(1 + \frac{1}{12 z} + R_2(z) \right)$$
where the remainder $R_2(z)$ obeys the bound
$$ |R_2(z)| \leq (2 \sqrt{2}+1) \frac{C_2 \Gamma(2)}{(2\pi)^3 |z|^2} $$
for $\mathrm{Re}(z) \geq 0$ and
$$|R_2(z)| \leq (2 \sqrt{2}+1) \frac{C_2 \Gamma(2)}{(2\pi)^3 |z|^2 |1 - e^{2\pi i z}|} $$
for $\mathrm{Re}(z) \leq 0$, where $C_2$ is the constant
$$ C_2 := \frac{1}{2} (1 + \zeta(2)) = \frac{1}{2} \left(1 + \frac{\pi^2}{6}\right).$$
In the latter case, we have $\mathrm{Im}(z) \geq 1$ by hypothesis, and hence $|1 - e^{2\pi i z}| \geq 1 - e^{-2\pi}$.  We conclude that in all ranges of $z$ of interest, we have
$$|R_2(z)| \leq (2 \sqrt{2}+1) \frac{C_2 \Gamma(2)}{(2\pi)^3 |z|^2 (1 - e^{-2\pi})} \leq \frac{0.0205}{|z|^2}$$
and hence by Claim (i)
$$ \Gamma(z) = \sqrt{2\pi} \exp\left( \left(z-\frac{1}{2}\right) \log z - z \right) \left(1 + O_{\leq}\left( \frac{1}{12(|z| - 0.246)} \right)\right) $$
and the claim then follows by Claim (ii).  

For Claim (vi), it suffices to show that the function $x \mapsto \frac{\log^a |x+iy|}{(x-c)^b}$ is non-increasing for $x \geq \exp(a/b)$.  Since $\log |x+iy| = (\log x) (1 + \frac{\log(1 + \frac{y^2}{x^2})}{2 \log x})$ and the second factor is monotone decreasing in $x$, it suffices to show that $x \mapsto \frac{\log^a x}{(x-c)^b}$ is non-increasing in this region.  Taking logarithms and differentiating, we wish to show that $\frac{a}{x \log x} - \frac{b}{x-c} \leq 0$.  But this is clear since $\frac{b}{x-c} \geq \frac{b}{x}$ and $\log x \geq a/b$.
\end{proof}
