\section{Initial estimation of $r_{t,n}, R_{t,N}$}

In this section we give some initial estimates on the functions $r_{t,n}, R_{t,N}$ appearing in Section \ref{heatflow-sec}.  

We begin with the estimation of $r_{t,n}$.  We will need the function
\begin{equation}\label{M-def}
 M_0(s) \coloneqq \frac{1}{8} \frac{s(s-1)}{2} \pi^{-s/2} \sqrt{2\pi} \exp( (\frac{s}{2}-\frac{1}{2})\log \frac{s}{2} - \frac{s}{2} )
\end{equation}
defined for all $s$ away from the negative axis.  Clearly this function is non-vanishing.  We may compute the logarithmic derivative $\alpha \coloneqq \frac{M'_0}{M_0}$ of this function as
\begin{equation}\label{alpha-form}
\begin{split}
 \alpha(s) &= \frac{1}{s} + \frac{1}{s-1} - \frac{1}{2} \log \pi + \frac{1}{2} \log \frac{s}{2} - \frac{1}{2s} \\
&= \frac{1}{2s} + \frac{1}{s-1} + \frac{1}{2} \log \frac{s}{2\pi}.
\end{split}
\end{equation}
We can also compute one further derivative:
\begin{equation}\label{alpha-deriv}
 \alpha'(s) = -\frac{1}{2s^2} - \frac{1}{(s-1)^2} + \frac{1}{2 s}.
\end{equation}
If $\mathrm{Im}(s) > 3$, we conclude in particular that
\begin{equation}\label{alpha-deriv-bound}
\begin{split}
 \alpha'(s) &= O_{\leq}( \frac{1}{2\Im(s)^2} ) + O_{\leq}( \frac{1}{\Im(s)^2} ) + O_{\leq}( \frac{1}{2\Im(s)} ) \\
 &= O_{\leq}( \frac{1}{2(\Im(s)-3)} )
\end{split}
\end{equation}
thanks to Lemma \ref{elem-lem}(i).  Finally, we introduce the more general functions
\begin{equation}\label{Mt-def}
M_t(s) \coloneqq \exp( \frac{t}{4} \alpha(s)^2 ) M_0(s)
\end{equation}
for any $t \geq 0$, as well as the sequence
$$ b_n(t) \coloneqq \exp( \frac{t}{4} \log^2 n ).$$

\begin{proposition}[Estimate for $r_{t,n}$]  Let $\sigma$ be real, let $T>10$, let $n$ be a positive integer, and let $0 < t \leq 1/2$.  Then 
$$ r_{t,n}(\sigma+iT) = M_t(\sigma+iT) \frac{b_n}{n^{\sigma+iT+\frac{t}{2} \alpha(\sigma+iT)}} (1 + O_{\leq}(\exp( \frac{1}{T-3.33} (\frac{t^2}{8} |\alpha(\sigma+iT) - \log n|^2 + \frac{t}{4} + \frac{1}{6}))-1))$$
\end{proposition}

\begin{proof}  From \eqref{ron-def}, \eqref{M-def} and Lemma \ref{elem-lem}(v) one has
$$ 
r_{0,n}(s) = M_0(s) n^{-s} \exp( O_{\leq}( \frac{1}{6(|s|-0.66)} ) )
$$
whenever $\mathrm{Im}(s) > 1$.  Set $\alpha_n \coloneqq \coloneqq \alpha(\sigma+iT) - \log n$ (this is the logarithmic derivative of $M(s) n^{-s}$ at $s=\sigma+iT$).  From \eqref{rtn-def} we have
$$
 r_{t,n}(\sigma+iT) = \exp( - \frac{t}{4} \alpha_n^2) \int_\R \exp( - \sqrt{t} v \alpha_n) M( \sigma + iT + \sqrt{t} v + \frac{t}{2} \alpha_n) 
\exp( O_{\leq}( \frac{1}{6(|\sigma+iT+\sqrt{t} v + \frac{t}{2} \alpha_n|-0.66)} ) )
\frac{1}{\sqrt{\pi}} e^{-v^2}\ dv.
$$
By \eqref{alpha-form} and the hypothesis $T \geq 10$, the imaginary part of $\alpha_n$ may be lower bounded by
$$ \mathrm{Im}(\alpha_n) \geq -\frac{1}{2T} - \frac{1}{T} \geq -0.15;$$
since $t \leq 1/2$, we conclude that $\sigma+iT+\sqrt{t} v + \frac{t}{2} \alpha_n$ has imaginary part at least $T - 0.08$.  Thus
$$
 r_{t,n}(s) = \exp( - \frac{t}{4} \alpha_n^2) \int_\R \exp( - \sqrt{t} v \alpha_n) M( \sigma + iT + \sqrt{t} v + \frac{t}{2} \alpha_n) \exp( -(\sigma+iT+\sqrt{t} v + \frac{t}{2} \alpha_n ) \log n + O_{\leq}( \frac{1}{6(T - 0.74)} ) )
\frac{1}{\sqrt{\pi}} e^{-v^2}\ dv.
$$
From \eqref{alpha-deriv-bound} we have
$$ \alpha'(s) = O_{\leq}( \frac{1}{2(T-3.08)} )$$
for all $s$ between $\sigma+iT$ and $\sigma + iT + \sqrt{t} v + \frac{t}{2} \alpha_n$.  Applying Taylor's theorem with remainder to a branch of the complex logarithm of $M_0$, we conclude that
$$ M_0( \sigma + iT + \sqrt{t} v + \frac{t}{2} \alpha_n) = M_0(\sigma+iT) \exp( \alpha(\sigma+iT) (\sqrt{t} v + \frac{t}{2} \alpha_n) + O_{\leq}( \frac{1}{4(T-3.08)} |\sqrt{t} v + \frac{t}{2} \alpha_n|^2 )).$$
Inserting this estimate, writing $\alpha(\sigma+iT) = \alpha_n + \log n$, estimating $\frac{1}{6(T-0.74)}$ by $\frac{1}{6(T-3.08)}$ and $|\sqrt{t} v + \frac{t}{2} \alpha_n|^2$ by $2tv^2 + \frac{t^2}{2} |\alpha_n|^2$, and simplifying, we conclude that
$$
 r_{t,n}(s) = M_0(\sigma+iT) \exp( \frac{t}{4} \alpha_n^2 - (\sigma+iT) \log n)
 \int_\R \exp(  O_{\leq}( \frac{1}{T-3.08} (\frac{t}{2}v^2 + \frac{t^2}{8} |\alpha_n|^2 + \frac{1}{6}) ) ) \frac{1}{\sqrt{\pi}} e^{-v^2}\ dv.$$
Writing $\alpha_n = \alpha(\sigma+iT) - \log n$ and using \eqref{Mt-def} we see that
$$ M_0(\sigma+iT) \exp( \frac{t}{4} \alpha_n^2 - (\sigma+iT) \log n) = M_t(\sigma+iT) \frac{b_n}{n^{\sigma+iT+\frac{t}{2} \alpha(\sigma+iT)}} $$
and so it suffices to show that
$$  \int_\R \exp(  O_{\leq}( \frac{1}{T-3.08} (\frac{t}{2}v^2 + \frac{t^2}{8} |\alpha_n|^2 + \frac{1}{6}) ) \frac{1}{\sqrt{\pi}} e^{-v^2}\ dv = 1 + O(\exp( \frac{1}{T-3.33} (\frac{t^2}{8} |\alpha_n|^2 + \frac{t}{4} + \frac{1}{6}))-1).$$
Since $\frac{1}{\sqrt{\pi}} e^{-v^2}\ dv $ integrates to one, it suffices by Lemma \ref{elem-lem}(iv) to show that
\begin{equation}\label{ax}
  \int_\R \exp( \frac{1}{T-3.08} (\frac{t}{2}v^2 + \frac{t^2}{8} |\alpha_n|^2 + \frac{1}{6})) \frac{1}{\sqrt{\pi}} e^{-v^2}\ dv \leq \exp( \frac{1}{T-3.33} (\frac{t^2}{8} |\alpha_n|^2 + \frac{t}{4} + \frac{1}{6})).
	\end{equation}
Using \eqref{gaussian}, the left-hand side may be calculated exactly as
$$ \exp( \frac{1}{T-3.08} (\frac{t^2}{8} |\alpha_n|^2 + \frac{1}{6})) (1 - \frac{t}{2(T-3.08)})^{-1/2}.$$
Applying Lemma \ref{elem-lem}(ii) and using the hypotheses $t \leq 1/2$, $T \geq 10$, one has
$$ 1 - \frac{t}{2(T-3.08)} = \exp( O_{\leq}( \frac{t}{2(T-3.33)} ))$$
and hence (bounding $\frac{1}{T-3.08}$ by $\frac{1}{T-3.33}$), we obtain the claim.
\end{proof}

Now we begin the estimation of $R_{t,N}$.  We begin with the following estimates of Arias de Reyna \cite{arias} on the term $\int_{N \swarrow N+1} \frac{w^{-s} e^{i\pi w^2}}{e^{\pi i w} - e^{-\pi i w}}$ appearing in \eqref{RON-def}:

\begin{proposition}\label{arias-prop}  Let $\sigma$ be real and $T'>0$, and define the quantities
\begin{align}
s &\coloneqq \sigma + iT' \label{s-def}\\
a &\coloneqq \sqrt{\frac{T'}{2\pi}} \label{a-def}\\
N &\coloneqq \lfloor a \rfloor \label{N-def}\\
p &\coloneqq 1 - 2(a-N) \label{p-def}\\
U &\coloneqq \exp( -i(\frac{T'}{2} \log \frac{T'}{2\pi} - \frac{T'}{2} - \frac{\pi}{8}) )\label{U-def}.
\end{align}
Let $K$ be a positive integer.  Then we have an expansion
$$ \int_{N \swarrow N+1} \frac{w^{-s} e^{i\pi w^2}}{e^{\pi i w} - e^{-\pi i w}} = (-1)^{N-1} U a^{-\sigma} (\sum_{k=0}^K \frac{C_k(p,\sigma)}{a^k} + RS_K(s)) $$
where $C_0(p,\sigma) = C_0(p)$ is independent of $\sigma$ and is given explicitly by
\begin{equation}\label{C0-def}
C_0(p)= \frac{e^{\pi i (\frac{p^2}{2} +\frac{3}{8})} - i \sqrt{2} \cos \frac{\pi p}{2}}{2 \cos(\pi p)}
\end{equation}
(removing the singularities at $p = \pm 1/2$), while for $k \geq 1$ the $C_k(p,\sigma)$ obey the bounds
\begin{equation}\label{ck-bound-1}
|C_k(p,\sigma)| \leq \frac{\sqrt{2}}{2\pi} \frac{9^\sigma \Gamma(k/2)}{2^k}
\end{equation}
for $\sigma>0$ and
\begin{equation}\label{ck-bound-2}
|C_k(p,\sigma)| \leq \frac{2^{\frac{1}{2}-\sigma}}{2\pi} \frac{\Gamma(k/2)}{2\pi ((3-2\log 2)\pi)^{k/2}}
\end{equation}
for $\sigma \leq 0$, while the error term $RS_K(s)$ obeys the bounds
\begin{equation}\label{rsk-bound-1}
|RS_K(s)| \leq \frac{1}{7} 2^{3\sigma/2} \frac{\Gamma((K+1)/2)}{(a/1.1)^{K+1}}
\end{equation}
for $\sigma \geq 0$, and
\begin{equation}\label{rsk-bound-2}
|RS_K(s)| \leq \frac{1}{2} (\frac{9}{10})^{\lceil -\sigma \rceil} \frac{\Gamma((K+1)/2)}{(a/1.1)^{K+1}}
\end{equation}
if $\sigma < 0$ and $K + \sigma \geq 2$.
\end{proposition}

\begin{proof} This follows from \cite[Theorems 3.1, 4.1, 4.2]{arias} combined with \cite[(3.2), (5.2)]{arias}.  The dependence of $C_k(p,\sigma), k \geq 1$ on $\sigma$ and the dependence of $RS_K(s)$ on $s$ is suppressed in \cite{arias}, but can be discerned from the definitions of these quantities (and the related quantities $g(\tau,z), P_k(z) = P_k(z,\sigma), Rg_K(\tau,z)$) in \cite[(3.9), (3.10), (3.7), (3.6)]{arias}.
\end{proof}

Note that $p$ ranges in the interval $[-1,1]$.  One can show that 
\begin{equation}\label{cop}
|C_0(p)| \leq \frac{1}{2}
\end{equation}
for all $p \in [-1,1]$; this follows for instance from the $n=0$ case of \cite[Theorem 6.1]{arias}.


\begin{proposition}[Estimate for $R_{t,N}$]  Let $0 \leq \sigma \leq 1$, let $T \geq 100$, and let $0 < t \leq 1/2$.  Set
$$ T' \coloneqq T + \frac{\pi t}{8} $$
and then define $a,N,p,U,C_0(p)$ using \eqref{a-def}, \eqref{N-def}, \eqref{U-def}, \eqref{C0-def}
Then 
$$
 R_{t,N}(\sigma+iT) = (-1)^{N-1} U e^{\pi i \sigma/4} \exp( \frac{t \pi^2}{64}) M_0(iT') ( C_0(p) + O_{\leq}( (\frac{0.366 \times 9^\sigma + 0.887}{a-0.125} + \frac{5}{3(T'-3.33)}) \exp( \frac{4.89}{T'-3.33} ) )).$$
\end{proposition}

\begin{proof}  We apply \eqref{RTN-def} with $\beta_N := \pi i/4$ to obtain
$$ R_{t,N}(\sigma+iT) = \exp( \frac{t \pi^2}{64}) \int_\R \exp( - \sqrt{t} v \pi i/4) R_{0,N}( \sigma+iT' + \sqrt{t} v) \frac{1}{\sqrt{\pi}} e^{-v^2}\ dv.$$
From \eqref{RON-def} we have
$$ R_{0,N}( \sigma+iT' + \sqrt{t} v) = \frac{1}{8} \frac{s_v(s_v-1)}{2} \pi^{-s_v/2} \Gamma(\frac{s_v}{2}) (-1)^{N-1} U a^{-\sigma-\sqrt{t} v}
(\sum_{k=0}^{K_v} \frac{C_k(p,\sigma + \sqrt{t} v)}{a^k} + RS_{K_v}(s_v)) $$
for any positive integer $K_v$ that we permit to depend on $v$, where $s_v \coloneqq \sigma + iT' + \sqrt{t} v$. From \eqref{M-def} and Lemma \ref{elem-lem}(v) we thus have
$$ R_{0,N}( \sigma+iT' + \sqrt{t} v) = M_0(s_v) \exp( O_{\leq}(\frac{1}{12(T'-0.33)}) ) (-1)^{N-1} U a^{-\sigma-\sqrt{t} v}
(\sum_{k=0}^K \frac{C_k(p,\sigma)}{a^k} + RS_K(s_v)).$$
From \eqref{alpha-deriv-bound} and Taylor expansion of a logarithm of $M$, we have
$$ M_0(s_v) = M_0(iT') \exp( \alpha(iT') (\sigma + \sqrt{t} v) + O_{\leq}( \frac{1}{4(T'-0.33)} (\sigma + \sqrt{t} v)^2 ) );$$
from \eqref{alpha-form}, \eqref{a-def} one has
$$ \alpha(iT') = O_{\leq}( \frac{1}{2T'}) + O_{\leq}( \frac{1}{T'}) + \frac{1}{2} \log \frac{iT'}{2\pi} = \log a + \frac{i\pi}{4} + O_{\leq}( \frac{3}{2T'}).$$
Bounding $\frac{3}{2T'}$ by $\frac{6}{4(T'-0.33)}$, we conclude that
$$ \exp( - \sqrt{t} v \pi i/4) R_{0,N}( \sigma+iT' + \sqrt{t} v) = 
M_0(iT') \exp( \frac{\pi i \sigma}{4} + O_{\leq}(\frac{1}{4(T'-0.33)} ((\sigma + \sqrt{t} v)^2+6|\sigma+\sqrt{t} v|+\frac{1}{3}) ) ) (-1)^{N-1} U (\sum_{k=0}^{K_v} \frac{C_k(p, \sigma+\sqrt{t} v)}{a^k} + RS_{K_v}(s_v)).$$
Bounding $6|\sigma+\sqrt{t} v| \leq 3 (\sigma + \sqrt{t} v)^2 + 3$, we have
$$ \frac{1}{4(T'-0.33)} ((\sigma + \sqrt{t} v)^2+6|\sigma+\sqrt{t} v|+\frac{1}{3}) \leq \frac{1}{T'-0.33} ((\sigma + \sqrt{t} v)^2 + \frac{5}{6});$$
putting all this together, we obtain
$$ R_{t,N}(\sigma+iT) = (-1)^{N-1} U e^{\pi i \sigma/4} \exp( \frac{t \pi^2}{64}) M_0(iT') \int_\R \exp( O_{\leq}(\frac{1}{T'-0.33} ((\sigma + \sqrt{t} v)^2 + \frac{5}{6}) ) ) (\sum_{k=0}^{K_v} \frac{C_k(p,\sigma + \sqrt{t} v)}{a^k} + RS_{K_v}(s_v)) \frac{1}{\sqrt{\pi}} e^{-v^2}\ dv.$$
We separate the $k=0$ term from the rest.
By Lemma \ref{elem-lem}(iv) and the fact that $\frac{1}{\sqrt{\pi}} e^{-v^2}$ integrates to one, we can write the above expression as
\begin{equation}\label{rtnst}
 R_{t,N}(\sigma+iT) = (-1)^{N-1} U e^{\pi i \sigma/4} \exp( \frac{t \pi^2}{64}) M_0(iT') ( C_0(p) (1 + O_{\leq}(\epsilon)) + O_{\leq}(\delta) )
\end{equation}
where
$$ \epsilon := \int_\R ( \exp( \frac{1}{T'-0.33} ((\sigma + \sqrt{t} v)^2 + \frac{5}{6}) )  - 1) \frac{1}{\sqrt{\pi}} e^{-v^2}\ dv$$
and
$$ \delta := \int_\R \exp( \frac{1}{T'-0.33} ((\sigma + \sqrt{t} v)^2 + \frac{5}{6}) ) (\sum_{k=1}^{K_v} \frac{|C_k(p,\sigma+\sqrt{t}v)|}{a^k} + |RS_{K_v}(s_v)|) \frac{1}{\sqrt{\pi}} e^{-v^2}\ dv.$$
Bounding $(\sigma + \sqrt{t} v)^2 \leq 2 \sigma^2 + 2 t v^2$ and using \eqref{gaussian} we obtain
$$ \epsilon \leq \exp( \frac{1}{T'-0.33} (2\sigma^2 + \frac{5}{6}) ) (1 - \frac{2t}{T'-0.33})^{-1/2} - 1.$$
Applying Lemma \ref{elem-lem}(ii) and using the hypotheses $t \leq 1/2$, $T \geq 100$, one has
$$ 1 - \frac{2t}{T'-0.33)} = \exp( O_{\leq}( \frac{2t}{T'-3.33} ))$$
and hence
$$
 \epsilon \leq \exp( \frac{1}{T'-3.33} (2\sigma^2 + t + \frac{5}{6}) ) - 1.
$$
With $t \leq 1/2$ and $0 \leq \sigma \leq 1$, one has $2\sigma^2 + t + \frac{5}{6} \leq \frac{10}{3}$.  By the mean value theorem we then have
\begin{equation}\label{eep}
 \epsilon \leq \frac{10}{3(T'-3.33)} \exp( \frac{10}{3(T'-3.33)}).
\end{equation}

Now we work on $\delta$.  Making the change of variables $u \coloneqq \sigma + \sqrt{t} v$, we have
$$ \delta = \int_\R \exp( \frac{1}{T'-0.33} (u^2 + \frac{5}{6}) ) (\sum_{k=1}^{\tilde K_u} \frac{|C_k(p,u)|}{a^k} + |RS_{\tilde K_u}(u + iT')|) \frac{1}{\sqrt{\pi t}} e^{-(u-\sigma)^2/t}\ du,$$
where $\tilde K_u$ is a positive integer parameter that can depend arbitrarily on $u$ (as long as it is measurable, of course).  

We choose $\tilde K_u$ to equal $1$ when $u \geq 0$ and $\lfloor -\sigma \rfloor + 3$ when $u < 0$, so that Proposition \ref{arias-prop} applies.  The expression
$$ \sum_{k=1}^{\tilde K_u} \frac{|C_k(p,u)|}{a^k} + |RS_{\tilde K_u}(u + iT')| $$
is then bounded by
\begin{equation}\label{u0}
 \frac{\sqrt{2}}{2\pi} \frac{9^u \Gamma(1/2)}{2a} + \frac{1}{7} 2^{3u/2} \frac{\Gamma(1)}{(a/1.1)^2}
\leq \frac{0.200 \times 9^u}{a} + \frac{0.173 \times 2^{3u/2}}{a^2} 
\end{equation}
for $u \geq 0$ and
\begin{equation}\label{laf}
 \sum_{1 \leq k \leq \lfloor -u \rfloor+3} \frac{2^{\frac{1}{2}-\sigma}}{2\pi} \frac{\Gamma(k/2)}{2\pi ((3-2\log 2)\pi)^{k/2} a^k} + \frac{1}{2} (9/10)^{\lceil -u \rceil} \frac{\Gamma((\lfloor -u \rfloor+4)/2)}{(a/1.1)^{\lfloor -u \rfloor+4}}
\end{equation}
for $u < 0$.  One easily verifies that
$$ \frac{2^{\frac{1}{2}-\sigma}}{2\pi} \frac{\Gamma(k/2)}{2\pi ((3-2\log 2)\pi)^{k/2} a^k}
\leq \frac{1}{2} (9/10)^{\lceil -\sigma \rceil} \frac{\Gamma(k/2)}{(a/1.1)^k} $$
and so we can bound \eqref{laf} by
$$ \frac{1}{2} (9/10)^{\lceil -u \rceil} \sum_{1 \leq k \leq -u+4} \frac{\Gamma(k/2)}{(a/1.1)^k}.$$

For $u \geq 0$, we can estimate \eqref{u0} by
$$ 0.2 \times 9^u (\frac{1}{a} + \frac{0.865}{a^2}) \leq \frac{0.2 \times 9^u}{a - 0.865}$$
thanks to Lemma \ref{elem-lem}(i).  For $u<0$, we observe that if $k \leq \frac{a^2}{1.21} = \frac{T'}{2.42 \pi}$ then
$$ \frac{\Gamma(k+2/2)}{(a/1.1)^{k+2}} = \frac{1.21 k}{2 a^2} \frac{\Gamma(k/2)}{(a/1.1)^k} \leq \frac{1}{2} \frac{\Gamma(k/2)}{(a/1.1)^k}$$
and hence by the geometric series formula
$$ \sum_{2 \leq k \leq \frac{T'}{2.42 \pi}, k\ \mathrm{even}} \frac{\Gamma(k/2)}{(a/1.1)^k}  \leq 2 \frac{\Gamma(2/2)}{(a/1.1)^2} = \frac{2.42}{a^2}$$
and similarly
$$ \sum_{3 \leq k \leq \frac{T'}{2.42 \pi}, k\ \mathrm{odd}} \frac{\Gamma(k/2)}{(a/1.1)^k}  \leq 2 \frac{\Gamma(3/2)}{(a/1.1)^3} = \frac{1.331 \sqrt{\pi}}{a^3}$$
and hence we can bound \eqref{laf} by
$$ \frac{1}{2} (9/10)^{\lceil -u \rceil} (\frac{1.1 \sqrt{\pi}}{a} + \frac{2.42}{a^2} + \frac{1.331 \sqrt{\pi}}{a^3} + \sum_{\frac{T'}{2.42 \pi} \leq k \leq -u+4} \frac{\Gamma(k/2)}{(a/1.1)^k} ).$$
By Lemma \ref{elem-lem}(i) we have
$$ \frac{1.1 \sqrt{\pi}}{a} + \frac{2.42}{a^2} + \frac{1.331 \sqrt{\pi}}{a^3} \leq \frac{1.1 \sqrt{\pi}}{a - 1.25}$$
and thus (bounding $(9/10)^{\lceil -u \rceil}$ by $1/(1.1)$) we can bound \eqref{laf} by
$$ \frac{1}{2} (\frac{\sqrt{\pi}}{a-1.25} + \sum_{\frac{T'}{2.2 \pi} \leq k \leq -u+4} (1.1)^{k-1} \frac{\Gamma(k/2)}{a^k} ).$$

Putting this together, we conclude that
$$
\sum_{k=1}^{\tilde K_u} \frac{|C_k(p,u)|}{a^k} + |RS_{\tilde K_u}(u + iT')| \leq 
\frac{0.2 \times 9^u}{a-0.865} + \frac{\sqrt{\pi}}{2(a - 1.25)} + \frac{1}{2} \sum_{\frac{T'}{2.2 \pi} \leq k \leq -u+4} (1.1)^{k-1} \frac{\Gamma(k/2)}{a^k}$$
for all $u$ (positive or negative).  We conclude that $\delta \leq \delta_1 + \delta_2 + \delta_3$, where
\begin{align*}
\delta_1 &\coloneqq \int_\R \exp( \frac{1}{T'-0.33} (u^2 + \frac{5}{6}) ) \frac{\sqrt{\pi}}{2(a - 1.25)} \frac{1}{\sqrt{\pi t}} e^{-(u-\sigma)^2/t}\ du\\
\delta_2 &\coloneqq \int_\R \exp( \frac{1}{T'-0.33} (u^2 + \frac{5}{6}) ) \frac{0.2 \times 9^u}{a-0.865} \frac{1}{\sqrt{\pi t}} e^{-(u-\sigma)^2/t}\ du\\
\delta_3 &\coloneqq \int_\R \exp( \frac{1}{T'-0.33} (u^2 + \frac{5}{6}) ) \frac{1}{2} \sum_{\frac{T'}{2.2 \pi} \leq k \leq -u+4} (1.1)^{k-1} \frac{\Gamma(k/2)}{a^k} \frac{1}{\sqrt{\pi t}} e^{-(u-\sigma)^2/t}\ du.
\end{align*}
Using \eqref{gaussian}, we may evaluate $\delta_1$ exactly as
$$ \delta_1 = \frac{\sqrt{\pi}}{2(a - 1.25)} \exp( \frac{5}{6(T'-0.33)} ) (1 - \frac{t}{T'-0.33})^{-1/2}.$$
From Lemma \ref{elem-lem}(ii) and the hypothesis $t \leq 1/2$ we have
\begin{equation}\label{el}
 (1 - \frac{t}{T'-0.33})^{-1/2} \leq \exp( \frac{t}{2(T'-0.83)} )
\end{equation}
and hence
$$ \delta_1 \leq \frac{\sqrt{\pi}}{2(a - 1.25)} \exp( \frac{5+3t}{6(T'-0.83)} ).$$

For $\delta_2$, we translate $u$ by $\sigma$ to obtain
$$ \delta_2 = \frac{0.2 \times 9^\sigma}{a-0.865} \int_\R \exp( \frac{1}{T'-0.33} (u^2 + 2 \sigma u + \sigma^2 + \frac{5}{6}) + 2 u \log 3 ) \frac{1}{\sqrt{\pi t}} e^{-u^2/t}\ du$$
and hence by \eqref{gaussian}
$$ \delta_2 = \frac{0.2 \times 9^\sigma}{a-0.865} \exp( \frac{\sigma^2 + \frac{5}{6}}{T'-0.33} + \frac{t(\log 3 + \frac{\sigma}{T'-0.33})^2}{1 - \frac{t}{T'-0.33}} ) (1 - \frac{t}{T'-0.33})^{-1/2}.$$
One can write
$$ \frac{1}{1 - \frac{t}{T'-0.33}} = 1 + \frac{t}{T'-0.33-t} \leq 1 + \frac{t}{T'-0.83}$$
and hence by \eqref{el}
$$ \delta_2 \leq \frac{0.2 \times 9^\sigma}{a-0.865} \exp( \frac{5+3t+6\sigma^2}{6(T'-0.83)} + t(\log 3 + \frac{\sigma}{T'-0.33})^2 (1 + \frac{t}{T'-0.83}) ).$$
From Lemma \ref{elem-lem}(i) and the hypothesis $0 \leq \sigma \leq 1$, we have
\begin{align*}
(\log 3 + \frac{\sigma}{T'-0.33})^2 &\leq \log^2 3 (1 + \frac{2 \sigma / \log 3}{T' - 0.33 - \frac{\sigma}{2\log 3}}) \\
&\leq  \log^2 3 (1 + \frac{2 \sigma / \log 3}{T' - 0.83})
\end{align*}
and then
\begin{align*}
(\log 3 + \frac{\sigma}{T'-0.33})^2 (1 + \frac{t}{T'-0.83}) 
&\leq \log^2 3 (1 + \frac{\frac{2 \sigma}{\log 3} + t}{T' - 0.83 - \frac{2\sigma t/\log 3}{2\sigma/\log 3 + t}}) \\
&\leq \log^2 3 (1 + \frac{\frac{2 \sigma}{\log 3} + t}{T' - 0.83 - t}) \\
&\leq \log^2 3 (1 + \frac{\frac{2 \sigma}{\log 3} + t}{T' - 1.33}) 
\end{align*}
and thus
$$ \delta_2 \leq \frac{0.2 \times 9^\sigma \exp( t \log^2 3 )}{a-0.865} \exp( \frac{5+3t+6\sigma^2 + 12 \sigma \log 3 + 6t \log^2 3}{6(T'-1.33)} ).$$
Now we turn to $\delta_3$. By the Fubini-Tonelli theorem, we have
$$ \delta_3 = \frac{1}{2 \sqrt{\pi t}} \sum_{k \geq \frac{T'}{2.2 \pi}} (1.1)^{k-1} \frac{\Gamma(k/2)}{a^k} \int_{-\infty}^{4-k} \exp( \frac{1}{T'-0.33} (u^2 + \frac{5}{6}) - \frac{(u-\sigma)^2}{t} )\ du.$$
Since $u \leq 4-k$, $k \geq \frac{T'}{2.2\pi}$, and $T' \geq T \geq 100$, we have $k \geq 14$ and $u \leq -10$; since $\sigma \geq 0$, we may thus lower bound $(u-\sigma)^2/t$ by $u^2/t$.  Since $t \leq 1/2$, we can upper bound $\frac{1}{T'-0.33} (u^2 + \frac{5}{6}) - \frac{u^2}{t}$ by $-\frac{u^2}{2t}$, thus
$$ \delta_3 \leq \frac{1}{2 \sqrt{\pi t}} \sum_{k \geq \frac{T'}{2.2 \pi}} (1.1)^{k-1} \frac{\Gamma(k/2)}{a^k} \int_{-\infty}^{4-k} e^{-u^2/2t}\ du.$$
We can bound $e^{-u^2} \leq e^{(k-4)u/2t}$, and thus
$$ \int_{-\infty}^{4-k} e^{-u^2/2t}\ du \leq \frac{2t}{k-4} e^{-(k-4)^2/2t} \leq \frac{2t}{k-4} e^{-(k-4)^2}$$
and thus
$$ \delta_3 \leq \frac{\sqrt{t}}{\sqrt{\pi}} \sum_{k \geq \frac{T'}{2.2 \pi}} (1.1)^{k-1} \frac{\Gamma(k/2)}{(k-4) a^k} e^{-(k-4)^2}.$$
For $k \geq 14$ one can easily verify that $(1.1)^{k-1} \Gamma(k/2) e^{-(k-4)^2} \leq 10^{-30}$; discarding the $\frac{\sqrt{t}}{\sqrt{\pi}}$ and $\frac{1}{k-4}$ factors we thus have
$$ \delta_3 \leq \sum_{k \geq 14} \frac{10^{-30}}{a^k} \leq \frac{2 \times 10^{-30}}{a^{14}}$$
(say).   Since
$$ \frac{0.2}{a-0.865} + \frac{2 \times 10^{-30}}{a^{14}} \leq \frac{0.2}{a-1.25}$$
we have
$$ \delta_2+\delta_3 \leq \frac{0.2 \times 9^\sigma \exp( t \log^2 3 )}{a-0.125} \exp( \frac{5+3t+6\sigma^2 + 12 \sigma \log 3 + 6t \log^2 3}{6(T'-1.33)} )$$
and therefore
$$ \delta \leq \frac{0.2 \times 9^\sigma \exp( t \log^2 3 ) + \frac{\sqrt{\pi}}{2}}{a-0.125} \exp( \frac{5+3t+6\sigma^2 + 12 \sigma \log 3 + 6t \log^2 3}{6(T'-1.33)} ).$$
With $t \leq 1/2$ and $0 \leq \sigma \leq 1$ one has
\begin{align*}
 0.2 \exp(t \log^2 3) &\leq 0.366 \\
\frac{\sqrt{\pi}}{2} &\leq 0.887 \\
\frac{5 + 3t + 6\sigma^2 + 12 \sigma \log 3 + 6t \log^2 3}{6} &\leq 4.89
\end{align*}
and hence
$$ \delta \leq \frac{0.366 \times 9^\sigma + 0.887}{a-0.125} \exp( \frac{4.89}{T'-1.33} ).$$
Inserting this and \eqref{eep}, \eqref{cop} into \eqref{rtnst} we obtain the claim.
\end{proof}

