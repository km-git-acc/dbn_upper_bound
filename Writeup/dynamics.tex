\section{Dynamics of zeroes}\label{dynamics-sec}

In this section we control the dynamics of the zeroes of $H_t$ in order to establish Theorem \ref{ubc-0}.  As $H_t$ is even with functional equation $H_t = H_t^*$, the zeroes are symmetric around the origin and the real axis; from \eqref{htdef} and the non-negativity of $\Phi$, we also see that $H_t(iy) > 0$ for all $y \in \R$, so there are no zeroes on the imaginary axis.  From the super-exponential decay of $\Phi$ and \eqref{htdef} we see that the entire function $H_t$ is of order $1$; by Jensen's formula, this implies that the number of zeroes in a large disk $D(0,R)$ is at most $O( R^{1+o(1)})$ as $R \to \infty$.

We begin with the analysis of the dynamics of a single zero of $H_t$:

\begin{proposition}[Dynamics of a single zero]\label{dynam}  Let $t_0 \in \R$, and let $(z_k(t_0))_{k \in \Z \backslash \{0\}}$ be an enumeration of the zeroes of $H_{t_0}$ in $\C$ (counting multiplicity), with the symmetry condition $z_{-k}(t_0) = -z_k(t_0)$.
\begin{itemize}
\item[(i)]  If $j \in \Z \backslash \{0\}$ is such that $z_j(t_0)$ is a simple zero of $H_{t_0}$, then there exists a neighbourhood $U$ of $z_j(t_0)$, a neighbourhood $I$ of $t_0$ in $\R$, and a smooth map $z_j: I \to U$ such that for every $t \in I$, $z_j(t)$ is the unique zero of $H_t$ in $U$.  Furthermore one has the equation
\begin{equation}\label{zjk}
 \frac{\partial z_j}{\partial t}(t_0) = 2 \sum^{\prime}_{k \neq j} \frac{1}{z_j(t_0) - z_k(t_0)} 
\end{equation}
where the sum is over those $k \in \Z \backslash \{0\}$ with $k \neq j$, and the prime means that the $k$ and $-k$ terms are summed together (except for the $k=-j$ term, which is summed separately) in order to make the sum convergent.
\item[(ii)]  If $j \in \Z \backslash \{0\}$ is such that $z_j(t_0)$ is a repeated zero of $H_{t_0}$ of order $m \geq 2$, then there is a neighbourhood $U$ of $z_j(t_0)$ such that for $t$ sufficiently close to $t_0$, there are precisely $m$ zeroes of $H_t$ in $U$, and they take the form
$$ z_j(t_0) + \sqrt{2} (t-t_0)^{1/2} x_j + O( |t-t_0|)$$
for $j=1,\dots,m$ as $t \to t_0$, where $x_1 < \dots < x_m$ are the roots of the $m^{\operatorname{th}}$ Hermite polynomial
\begin{align}
\operatorname{He}_m(z) &\coloneqq (-1)^m \exp\left(\frac{z^2}{2}\right) \frac{d^m}{dz^m} \exp\left(-\frac{z^2}{2}\right)\label{heform}\\
&= \sum_{0 \leq l \leq m/2} \frac{m!}{l! (m-2l)!} (-1)^l \frac{z^{m-2l}}{2^l}\label{heform2}
\end{align}
and the implied constant in the $O()$ notation can depend on $t_0$, $j$, and $m$.
\end{itemize}
\end{proposition}

The differential equation \eqref{zjk} was previously derived in \cite[Lemma 2.4]{csv} in the case $t > \Lambda$ (in which all zeroes are real and simple).
The $x_1,\dots,x_m$ can be given explicitly for small values of $m$ as
$$ x_1 = -1; \quad x_2 = +1$$
when $m=2$,
$$ x_1 = -\sqrt{3}; \quad x_2 = 0; \quad x_3 = +\sqrt{3}$$
when $m=3$, and
$$ x_1 = -\sqrt{3+\sqrt{6}}; \quad x_2 = -\sqrt{3-\sqrt{6}}; \quad x_3 = \sqrt{3 - \sqrt{6}}; \quad  x_4 = \sqrt{3 + \sqrt{6}}$$
when $m=4$.  From \eqref{heform} and iterating Rolle's theorem we see that all the zeroes $x_1,\dots,x_m$ of $\operatorname{He}_m$ are real; from the Hermite equation $\left(\frac{d^2}{dz^2} - z \frac{d}{dz} + m\right) \operatorname{He}_m(z) = 0$ and the Picard uniqueness theorem for ODE we see that the zeroes are all simple.

\begin{proof}  First suppose we are in the situation of (i).  As $z_j(t_0)$ is simple, $\frac{\partial}{\partial z} H_t$ is non-zero at $z_j(t_0)$; since $H_t(z)$ is a smooth function of both $t$ and $z$, we conclude from the implicit function theorem that there is a unique solution $z_j(t) \in U$ to the equation
$$ H_t( z_j(t) ) = 0$$
with $z_j(t)$ in a sufficiently small neighbourhood $U$ of $z_j(t_0)$, if $t$ is in a sufficiently small neighbourhood $I$ of $t_0$; furthermore, $z_j(t)$ depends smoothly on $t$, and agrees with $z_j(t_0)$ when $t=t_0$.  Differentiating the above equation at $t_0$, we obtain
$$ \frac{\partial H_t}{\partial t}|_{t=t_0}( z_j(t_0) ) + \frac{\partial z_j}{\partial t}(t_0) H'_{t_0}(z_j(t_0)) = 0,$$
where the primes denote differentiation in the $z$ variable.
On the other hand, from \eqref{htdef} and differentiation under the integral sign (which can be justified using the rapid decrease of $\Phi$) we have the backwards heat equation
\begin{equation}\label{back}
\frac{\partial H_t}{\partial t} = -H''_t
\end{equation}
for all $t \geq 0$.  Inserting this into the previous equation, we conclude that
\begin{equation}\label{tzj}
\frac{\partial z_j}{\partial t}(t_0)  = \frac{H''_t}{H'_t}( z_j(t_0) ),
\end{equation}
noting that the denominator $H'_t(z_j(t_0))$ vanishes by the hypothesis that the zero at $z_j(t_0)$ is simple.  Henceforth we omit the dependence on $t_0$ for brevity.  From Taylor expansion of $H_t$, $H'_t$, and $H''_t$ around the simple zero $z_j$ we see that
\begin{equation}\label{htz-eq}
 \frac{H''_t}{H'_t}( z_j) = 2 \lim_{z \to z_j}\left(  \frac{H'_t}{H_t}( z_j) - \frac{1}{z-z_j} \right).
\end{equation}
On the other hand, as $H_t$ is even, non-zero at the origin, and entire of order $1$, we see from the Hadamard factorization theorem that
$$ H_t(z) = H_t(0) \prod_k^{\prime} \left(1 - \frac{z}{z_k}\right),$$
where the prime indicates that the $k$ and $-k$ factors are multiplied together.  Note that the product converges absolutely since the number of zeroes in $D(0,R)$ grows like $O( R^{1+o(1)})$.  Taking logarithmic derivatives, we conclude that
$$ \frac{H'_t}{H_t}(z) = \sum_k^{\prime} \frac{1}{z-z_k}.$$
Inserting this into \eqref{tzj}, \eqref{htz-eq} and using the dominated convergence theorem (again using the growth in the number of zeroes to justify the interchange of summation and limits), we obtain the claim (i).

Now we prove (ii).  We abbreviate $z_j(t_0)$ as $z_j$.  By Taylor expansion we have
$$ \frac{\partial^{2k} H_{t_0}}{\partial z^{2k}}(z) = m (m-1) \dots (m-2k+1) a_m (z-z_j)^{m-2k} + O( |z-z_j|^{\max(m-2k+1,0)} )$$
as $z \to z_j$ for any fixed integer $k \geq 0$ and some non-zero complex number $a_m = a_m(z_j, t_0)$ (with the implied constant in the $O()$ notation allowed to depend on $k$, $z_j$, $t_0$); applying the backwards heat equation \eqref{back} we thus have
$$ \frac{\partial^k H_t}{\partial t^k}|_{t=t_0}(z) = (-1)^k m (m-1) \dots (m-2k+1) a_m (z-z_j)^{m-2k} + O( |z-z_j|^{\max(m-2k+1,0)} ).$$
Performing Taylor expansion in time and using \eqref{heform2}, we conclude that in the regime $z - z_j = O( |t-t_0|^{1/2} )$, one has the bound
$$ H_t(z) = a_m ((t-t_0)^{1/2})^m \left( \operatorname{He}_m\left( \sqrt{2} \frac{z-z_j}{(t-t_0)^{1/2}} \right) + O\left( |t-t_0|^{1/2} \right) \right)$$
as $t \to t_0$, using (say) the standard branch of the square root.  By the inverse function theorem (and the simple nature of the zeroes of $\operatorname{He}_m$), we conclude that for $t$ sufficiently close but not equal to to $t_0$, we have $m$ zeroes of $H_t$ of the form
$$ z_j + (t-t_0)^{1/2} x_j + O( |t-t_0| ).$$
By Rouche's theorem, if $U$ is a sufficiently small neighborhood of $z_j$ then these are the only zeroes of $H_t$ in $U$ for $t$ sufficiently close to $t_0$.  The claim follows.
\end{proof}

Next, we recall the following bound of de Bruijn:

\begin{theorem}\label{debr-bound}  Suppose that $t_0 \in \R$ and $y_0 > 0$ is such that there are no zeroes $H_{t_0}(x+iy)=0$ with $x \in \R$ and $y > y_0$.  Then for any $t>t_0$, there are no zeroes $H_{t}(x+iy)=0$ with $x \in \R$ and $y > \max( y_0^2 - 2(t-t_0), 0)^{1/2}$.  In particular one has $\Lambda \leq t_0 + \frac{1}{2} y_0^2$.
\end{theorem}

\begin{proof} See \cite[Theorem 13]{debr}.
\end{proof}

We are now ready to prove Theorem \ref{ubc-0}.  The main step is to establish

\begin{proposition}[Zero-free region criterion]\label{ubc}  Suppose that $t_0, X > 0$ and $0 < y_0 \leq 1$ obey the following hypotheses:
\begin{itemize}
\item[(i)]  There are no zeroes $H_0(x+iy)=0$ with $0 \leq x \leq X$ and $\sqrt{y_0^2+2 t_0} \leq y \leq 1$.
\item[(ii)]  There are no zeroes $H_{t_0}(x+iy)=0$ with $x \geq X+\sqrt{1-y_0^2}$ and $y_0 \leq y \leq \sqrt{1-2t_0}$.
\item[(iii)]  There are no zeroes $H_{t}(x+iy)=0$ with $X \leq x \leq X+\sqrt{1-y_0^2}$, $\sqrt{y_0^2 + 2(t_0-t)} \leq y \leq \sqrt{1-2t}$, and $0 \leq t \leq t_0$.
\end{itemize}
Then there are no zeroes $H_{t_0}(x+iy) = 0$ with $x \in \R$ and $y \geq y_0$.
\end{proposition}

\begin{proof}  It is well known that the Riemann $\xi$ function has no zeroes outside of the strip $\{ 0 \leq \mathrm{Re}(s) \leq 1 \}$, hence there are no zeroes $H_0(x+iy)=0$ with $y > 1$.  By Theorem \ref{debr-bound}, we may thus remove the upper bound constraints $y \leq 1$, $y \leq \sqrt{1-2t_0}$, and $y \leq \sqrt{1-2t}$ from (i), (ii), and (iii) respectively.

By hypotheses (ii), (iii) and the symmetry properties of $H_t$, it suffices to show that for every $0 \leq t \leq t_0$, there are no zeroes $H_t(x+iy) = 0$ with $0 \leq x \leq X$ and $y \geq Y(t)$, where $Y(t) \coloneqq \sqrt{y_0^2 + 2(t_0-t)}$.  By hypothesis (i), this is true at time $t=0$.  Suppose the claim failed for some time $0 < t \leq t_0$.  Let $t_1 \in (0,t_0]$ be the minimal time in which this occurred (such a time exists because $H_t$ varies continuously in $t$, and there are no zeroes $H_t(x+iy)=0$ with (say) $y>1$).  From Rouche's theorem (or Proposition \ref{dynam}) we conclude that there is a zero $H_{t_1}(x+iy)=0$ with $x+iy$ on the boundary of the region $\{ x+iy: 0 \leq	 x \leq X, y \geq Y(t_1) \}$.  The right side $x=X$ of this boundary is ruled out by hypothesis (ii), and (as mentioned at the start of the section) the left side $x=0$ is ruled out by \eqref{htdef} and the positivity of $\Phi$.  Thus by the symmetry properties of $H_{t_1}$ we must have
$$ H_{t_1}(x+iY(t_1)) = 0$$
for some $0 < x < X$.

Suppose first that $H_{t_1}$ has a repeated zero at $x+iy_0$.  Using Proposition \ref{dynam}(ii) and observing (from the symmetry of $\operatorname{He}_m$) that at least one of the roots $x_1,\dots,x_m$ is positive, we then see that for $t<t_1$ sufficiently close to $t_1$, $H_t$ has a zero in the region $\{ x+iy: 0 \leq x \leq X, y \geq Y(t) \}$, contradicting the minimality of $t_1$.  Thus the zero $x+i Y(t_1)$ of $H_{t_1}$ must be simple.  In particular, by Proposition \ref{dynam}(i) we can write $x+i Y(t_1) = z_j(t_1)$ for some smooth function $z_j$ in a neighbourhood of $t_1$ obeying \eqref{zjk}, such that $z_j(t)$ is a zero of $H_t$ for all $t$ close to $t_1$.  We will prove that
\begin{equation}\label{im}
\mathrm{Im} \frac{\partial}{\partial t} z_j( t_1 ) < \frac{\partial}{\partial t} Y(t_1),
\end{equation}
which implies that there is a zero of $H_t$ in the region $\{ x+iy: 0 \leq x \leq X, y \geq Y(t) \}$  for $t<t_1$ sufficiently close to $t_1$, giving the required contradiction.  

The right-hand side of \eqref{im} is
$$ \frac{\partial}{\partial t} Y(t_1) = -\frac{1}{Y(t_1)}.$$
By Proposition \ref{dynam}(i), the left-hand side of \eqref{im} is
$$ 2 \sum_{k \neq j}^{\prime} \frac{Y(t_1) - y_k}{(x-x_k)^2 + (Y(t_1)-y_k)^2}$$
where we write $z_k = x_k + i y_k$.  Clearly any zero $x_k+iy_k$ with imaginary part $y_k$ in $[-Y(t_1),Y(t_1)]$ gives a non-positive contribution to this sum, the contribution of the zero $x- iY(t_1)$ is $-\frac{1}{Y(t_1)}$, the contribution of the zero $-x+iY(t_1)$ vanishes, and the contribution of $-x-iY(t_1)$ is negative.  Grouping the remaining zeroes with their complex conjugates, it then suffices to show that
$$ \frac{Y(t_1) - y_k}{(x-x_k)^2 + (y_j-y_k)^2} + \frac{Y(t_1) + y_k}{(x-x_k)^2 + (Y(t_1)+y_k)^2} \leq 0$$
whenever $y_k > Y(t_1)$.  Cross-multiplying and canceling like terms, this inequality eventually simplifies to
$$ y_k^2 \leq (x-x_k)^2 + Y(t_1)^2.$$
But from the hypothesis (iii) and the assumption $y_k > Y(t_1)$, we have $|x_k| \geq X+\sqrt{1-Y(t_1)^2}$, so $(x-x_k)^2 \geq 1-Y(t_1)^2$.  On the other hand from Theorem \ref{debr-bound} one has $y_k < 1$, giving the required contradiction.
\end{proof}

By combining Proposition \ref{ubc} with Theorem \ref{debr-bound}, we obtain Theorem \ref{ubc-0}, noting from \eqref{hoz}, \eqref{sas} that condition (i) of Proposition \ref{ubc} is equivalent to condition (i) of Theorem \ref{debr-bound}.
