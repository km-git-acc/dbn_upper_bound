\section{Further numerical results}\label{further-sec}

By Theorem \ref{Zero}, one can verify the second hypothesis of Theorem \ref{ubc-0} when $X \geq \exp(C/t_0)$ for a large constant $C$.  If we ignore for sake of discussion the third hypothesis of Theorem \ref{ubc-0} (which turns out to be relatively easy to verify numerically in practice), this suggests that one can obtain a bound of the form $\Lambda \leq O(t_0)$ provided that one can verify the Riemann hypothesis up to a height $\exp(C/t_0)$.  In other words, if one has numerically verified the Riemann hypothesis up to a large height $T$, this should soon lead to a bound of the form $\Lambda \leq O \left( \frac{1}{\log T} \right )$.

Aside from improving the implied constant in this bound, it does not seem easy to improve this sort of implication without a major breakthrough on the Riemann hypothesis (such as a massive expansion of the known zero-free regions for the zeta function inside the critical strip).  We shall justify this claim heuristically as follows. Suppose that there was a counterexample to the Riemann hypothesis at a large height $T$, so that $H_0(2T + iy) = 0$ for some positive $y$, which for this discussion we will take to be comparable to $1$.  The Riemann von Mangoldt formula indicates that the number of zeroes of $H_0$ within a bounded distance of this zero should be comparable to $\log T$; the majority of these zeroes should obey the Riemann hypothesis and thus stay at roughly unit distance from our initial zero $2T+iy$.  Proposition \ref{dynam} then suggests that as time $t$ advances, this zero should move at speed comparable to $\log T$.  Thus one should not expect this zero to reach the real axis until a time comparable to $\frac{1}{\log T}$.  This heuristic analysis therefore indicates that it is unlikely that one can significantly improve the bound $\Lambda \leq O \left( \frac{1}{\log T} \right )$ without being able to exclude significant violations of the Riemann hypothesis at height $T$.

In this section we collect some numerical results verifying the second two hypotheses of Theorem \ref{ubc-0} for larger values of $X$, and smaller values of $t_0$, than were considered in Section \ref{newup-sec}.  This leads to improvements to the bound $\Lambda \leq 0.22$ conditional on the assumption that the Riemann Hypothesis can be numerically verified beyond the height $T \approx 1.2 \times 10^{11}$ used in Section \ref{newup-sec}.  Our conclusions are summarised in the table below, with the Moll2 bound describing the lower bound on $|f_t(x+iy)|$ obtained by using the Euler mollifier $E_{t,2}(x+iy)$, and $N_a$ the initial value of $N$ at the point $X$.

\begin{table}[h!]
  \begin{center}
    \caption{Conditional $\Lambda$ Results}
    \label{tab:table1}
    \begin{tabular}{l|r|r|r|c|r|c} % <-- Alignments: 1st column left, 2nd middle and 3rd right, with vertical lines in between
      $X$ & $t_{0}$ & $y_{0}$ & $\Lambda$ & $\textbf{Winding Number}$ & $N_{a}$ & $\textbf{Moll2 Bound}$\\
      \hline
      $2 \times 10^{12} + 129093$ & 0.198 & 0.15492 & 0.21 & 0 & 398942 & 0.0341\\
      $5 \times 10^{12} + 194858$ & 0.186 & 0.16733 & 0.20 & 0 & 630783 & 0.0376\\
      $2 \times 10^{13} + 131252$ & 0.180 & 0.14142 & 0.19 & 0 & 1261566 & 0.0349\\
      $6 \times 10^{13} + 123375$ & 0.168 & 0.15492 & 0.18 & 0 & 2185096 & 0.0377\\
      $3 \times 10^{14} + 188911$ & 0.161 & 0.13416 & 0.17 & 0 & 4886025 & 0.0369\\
      $2 \times 10^{15} + 122014$ & 0.153 & 0.11832 & 0.16 & 0 & 12615662 & 0.0532\\
      $7 \times 10^{15} + 68886$ & 0.139 & 0.14832 & 0.15 & 0 & 23601743 & 0.0350\\
      $6 \times 10^{16} + 156984$ & 0.132 & 0.12649 & 0.14 & 0 & 69098829 & 0.0307\\
      $6 \times 10^{17} + 88525$ & 0.122 & 0.12649 & 0.13 & 0 & 218509686 & 0.0347\\
      $9 \times 10^{18} + 35785$ & 0.113 & 0.11832 & 0.12 & 0 & 846284375 & 0.0318\\
      $2 \times 10^{20} + 66447$ & 0.102 & 0.12649 & 0.11 & 0 & 3989422804 & 0.0305\\
      $9 \times 10^{21} + 70686$ & 0.093 & 0.11832 & 0.1 & 0 & 26761861742 & 0.0321\\
    \end{tabular}
  \end{center}
\end{table}

The computations for $X=2 \times 10^{20} + 66447$ and $X=9 \times 10^{21} + 70686$ in the above table were massive, and performed using a Boinc based grid computing setup, in which a few hundred volunteers participated. Their contributions can be tracked at {\tt anthgrid.com/dbnupperbound}.
