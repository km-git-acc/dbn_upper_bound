\section{Further numerical results}

By Theorem \ref{Zero}, one can verify the second hypothesis of Theorem \ref{ubc-0} when $X \geq \exp(C/t_0)$ for a large constant $C$.  If we ignore for sake of discussion the third hypothesis of Theorem \ref{ubc-0} (which turns out to be relatively easy to verify numerically in practice), this suggests that one can obtain a bound of the form $\Lambda \leq O(t_0)$ provided that one can verify the Riemann hypothesis up to a height $\exp(C/t_0)$.  In other words, if one has numerically verified the Riemann hypothesis up to a large height $T$, this should soon lead to a bound of the form $\Lambda \leq O \left( \frac{1}{\log T} \right )$.

Aside from improving the implied constant in this bound, it does not seem easy to improve this sort of implication without a major breakthrough on the Riemann hypothesis (such as a massive expansion of the known zero-free regions for the zeta function inside the critical strip).  We shall justify this claim heuristically as follows. Suppose that there was a counterexample to the Riemann hypothesis at a large height $T$, so that $H_0(2T + iy) = 0$ for some positive $y$, which for this discussion we will take to be comparable to $1$.  The Riemann von Mangoldt formula indicates that the number of zeroes of $H_0$ within a bounded distance of this zero should be comparable to $\log T$; the majority of these zeroes should obey the Riemann hypothesis and thus stay at roughly unit distance from our initial zero $2T+iy$.  Proposition \ref{dynam} then suggests that as time $t$ advances, this zero should move at speed comparable to $\log T$.  Thus one should not expect this zero to reach the real axis until a time comparable to $\frac{1}{\log T}$.  This heuristic analysis therefore indicates that it is unlikely that one can significantly improve the bound $\Lambda \leq O \left( \frac{1}{\log T} \right )$ without being able to exclude significant violations of the Riemann hypothesis at height $T$.

In this section we collect some numerical results verifying the second two hypotheses of Theorem \ref{ubc-0} for larger values of $X$, and smaller values of $t_0$, than were considered in Section \ref{newup-sec}.  This leads to improvements to the bound $\Lambda \leq 0.22$ conditional on the assumption that the Riemann Hypothesis can be numerically verified beyond the height $T \approx 1.2 \times 10^{11}$ used in Section \ref{newup-sec}.  Our conclusions may be summarised as follows:

...
