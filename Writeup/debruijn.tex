
%\pdfoutput=1
%\documentclass[fms,times]{cuparticle}
\documentclass[a4paper,11pt,twoside]{amsart}
\usepackage{amsmath}
\usepackage{geometry}
%\newproof{proof}{Proof}
\usepackage{amssymb, latexsym}
\usepackage{verbatim}
%\usepackage{amscd}
\usepackage{enumerate}
\usepackage{comment}
\usepackage{txfonts}
%\usepackage[breaklinks=true,hidelinks]{hyperref}

%\usepackage{psfig}
%\usepackage{graphicx}
%\usepackage{showkeys}  
%\usepackage{siunitx}
%\usepackage{tikz-cd}
%\usepackage{color}
%\usetikzlibrary{arrows}
% uncomment this when editing cross-references
%\numberwithin{equation}{section}

%\volume{}
%\doi{}


% \usepackage{mathabx}


\usepackage{mathtools}%                  http://www.ctan.org/pkg/mathtools
\usepackage[tableposition=top]{caption}% http://www.ctan.org/pkg/caption
\usepackage{booktabs,dcolumn}%           http://www.ctan.org/pkg/dcolumn + http://www.ctan.org/pkg/booktabs

     
%\theoremstyle{plain}

\newtheorem{theorem}{Theorem}[section]
%\newtheorem{theorem}[theorem]{Theorem}
\newtheorem{proposition}[theorem]{Proposition}
\newtheorem{hypothesis}[theorem]{Hypothesis}
\newtheorem{lemma}[theorem]{Lemma}
\newtheorem{corollary}[theorem]{Corollary}
\newtheorem{conjecture}[theorem]{Conjecture}
\newtheorem{principle}[theorem]{Principle}
\newtheorem{claim}[theorem]{Claim}

%\theoremstyle{definition}

%\newtheorem{roughdef}[subsection]{Rough Definition}
\newtheorem{definition}[theorem]{Definition}
\newtheorem{remark}[theorem]{Remark}
\newtheorem{remarks}[theorem]{Remarks}
\newtheorem{example}[theorem]{Example}
\newtheorem{examples}[theorem]{Examples}
%\newtheorem{problem}[subsection]{Problem}
%\newtheorem{question}[subsection]{Question}

\newcommand\F{\mathbb{F}}
\newcommand\E{\mathbb{E}}
\newcommand\R{\mathbb{R}}
\newcommand\Z{\mathbb{Z}}
\newcommand\N{\mathbb{N}}
\newcommand\D{\mathbb{D}}
\newcommand\C{\mathbb{C}}
\newcommand\Q{\mathbb{Q}}
\newcommand\T{\mathbb{T}}
\renewcommand\Re{{\operatorname{Re\,}}}
\renewcommand\Im{{\operatorname{Im\,}}}
\newcommand\eps{\varepsilon}



\renewcommand\P{\mathbf{P}}


%%%%%%%%%%%%%%%%%%%%%%%%%%%%%%%%


\newcommand{\alert}[1]{{\bf \color{red} TODO: #1}}


\setlength\evensidemargin\oddsidemargin
%\setlength{\parindent}{0cm}

\begin{document}
\title[Upper bound for de Bruijn-Newman constant]{Effective approximation of heat flow evolution of the Riemann xi function, and an upper bound for the de Bruijn-Newman constant}

\author{D.H.J. Polymath}
\address{\tt{http://michaelnielsen.org/polymath1/index.php}}


\begin{abstract}
For each $t \in \R$, define the entire function
$$ H_t(z) \coloneqq \int_0^\infty e^{tu^2} \Phi(u) \cos(zu)\ du$$
where $\Phi$ is the super-exponentially decaying function
$$ \Phi(u) \coloneqq \sum_{n=1}^\infty (2\pi^2  n^4 e^{9u} - 3\pi n^2 e^{5u} ) \exp(-\pi n^2 e^{4u} ).$$
Newman showed that there exists a finite constant $\Lambda$ (the \emph{de Bruijn-Newman constant}) such that the zeroes of $H_t$ are all real precisely when $t \geq \Lambda$.  The Riemann hypothesis is the equivalent to the assertion $\Lambda \leq 0$, and Newman conjectured the complementary bound $\Lambda \geq 0$.


\end{abstract}


\maketitle

\input intro
\input heatflow
\input elementary
\input initial
\input dirichlet
\input lowx
\input zerofree
 %\appendix
%\input appendix



\begin{thebibliography}{99} 

\bibitem{arias}
J. Arias de Reyna, \emph{High-precision computation of Riemann's zeta function by the Riemann-Siegel asymptotic formula, I}, Mathematics of Computation, Volume 80, Number 274, April 2011, Pages 995–1009.

\bibitem{boyd}
W. G. C. Boyd, \emph{Gamma Function Asymptotics by an Extension of the Method of Steepest Descents}, Proceedings: Mathematical and Physical Sciences, Vol. 447, No. 1931 (Dec. 8, 1994), pp. 609--630.

\bibitem{debr}
N. C. de Bruijn, \emph{The roots of trigonometric integrals}, Duke J. Math. \textbf{17} (1950), 197--226.

%\bibitem{cmmrs}
%A. Chang, D. Mehrle, S. J. Miller, T. Reiter, J. Stahl, D. Yott, \emph{Newman's conjecture in function fields}, J. Numb. Thy. \textbf{157} (2015), 154--169.
%

\bibitem{cnv}
G. Csordas, T. S. Norfolk, R. S. Varga, \emph{A lower bound for the de Bruijn-Newman constant $\Lambda$}, Numer. Math. \textbf{52} (1988), 483--497.

\bibitem{cosv}
G. Csordas, A. M. Odlyzko, W. Smith, R. S. Varga, \emph{A new Lehmer pair of zeros and a new lower bound for the De Bruijn-Newman constant Lambda}, Electronic Transactions on Numerical Analysis. \textbf{1} (1993), 104--111.

\bibitem{crv}
G. Csordas, A. Ruttan, R.S. Varga, \emph{The Laguerre inequalities with applications
to a problem associated with the Riemann hypothesis}, Numer. Algorithms, \textbf{1} (1991), 305--329.

\bibitem{csv}
G. Csordas, W. Smith, R. S. Varga, \emph{Lehmer pairs of zeros, the de Bruijn-Newman constant $\Lambda$, and the Riemann hypothesis},  Constr. Approx. \textbf{10} (1994), no. 1, 107--129. 

\bibitem{kkl}
H. Ki, Y. O. Kim, and J. Lee, \emph{On the de Bruijn-Newman constant}, Advances in Mathematics, \textbf{22} (2009), 281--306.

\bibitem{lehmer}
D. H. Lehmer, \emph{On the roots of the Riemann zeta-function}, Acta Math. \textbf{95} (1956) 291--298.

\bibitem{montgomery}
H. L. Montgomery, \emph{The pair correlation of zeros of the zeta function}, Analytic number theory (Proc. Sympos. Pure Math., Vol. XXIV, St. Louis Univ., St. Louis, Mo., 1972), pp. 181--193. Amer. Math. Soc., Providence, R.I., 1973.

\bibitem{MoVa}
H. L. Montgomery, R. C. Vaughan, Multiplicative number theory. I. Classical theory. Cambridge Studies in Advanced Mathematics, 97. Cambridge University Press, Cambridge, 2007.

\bibitem{newman}
C. M. Newman, \emph{Fourier transforms with only real zeroes}, Proc. Amer. Math. Soc. \textbf{61} (1976), 246--251.

\bibitem{nrv}
T. S. Norfolk, A. Ruttan, R. S. Varga, \emph{A lower bound for the de Bruijn-Newman
constant $\Lambda$ II.}, in A. A. Gonchar and E. B. Saff, editors, Progress in Approximation
Theory, 403--418. Springer-Verlag, 1992.

\bibitem{odlyzko}
A. M. Odlyzko, \emph{An improved bound for the de Bruijn-Newman constant}, Numerical Algorithms \textbf{25} (2000), 293--303.

\bibitem{brad}
B. Rodgers, T. Tao, \emph{The De Bruijn-Newman constant is nonnegative}, preprint.

\bibitem{saouter}
Y. Saouter, X. Gourdon, P. Demichel, \emph{An improved lower bound for the de Bruijn-Newman constant}, Mathematics of Computation. \textbf{80} (2011), 2281--2287. 

\bibitem{stopple}
J. Stopple, \emph{Notes on Low discriminants and the generalized Newman conjecture}, Funct. Approx. Comment.
Math., vol. 51, no. 1 (2014), pp. 23--41.

\bibitem{stopple-2}
J. Stopple, \emph{Lehmer pairs revisited}, Exp. Math. \textbf{26} (2017), no. 1, 45--53. 

\bibitem{tr}
H. J. J. te Riele, \emph{A new lower bound for the de Bruijn-Newman constant}, Numer. Math., \textbf{58} (1991), 661--667.

\bibitem{titch}
E. C. Titchmarsh, The Theory of the Riemann Zeta-function, Second ed. (revised by D. R. Heath-Brown), Oxford University Press, Oxford, 1986.	


\end{thebibliography} 
























\end{document}
